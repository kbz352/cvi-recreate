\documentclass[10pt, letterpaper]{article}
\usepackage[utf8]{inputenc}
\usepackage{amsmath, amssymb}
\usepackage{graphicx}

%\usepackage{microtype} % Slightly tweak font spacing for aesthetics
\usepackage[
    protrusion=true,
    activate={true,nocompatibility},
    final,
    tracking=true,
    kerning=true,
    spacing=true,
    factor=1100]{microtype}
\SetTracking{encoding={*}, shape=sc}{40}

\usepackage[
    % margin=1in
    top=1in,
    bottom=0.5in,
    left=1in,
    right=1in,
    % showframe % Uncomment to show frames around the margins for debugging purposes
]{geometry}
\usepackage[bottom, hang]{footmisc} % Force footnotes to the bottom of the page and to the left margin
\setlength{\footnotemargin}{6pt} % Horizontal space between the footnote marker and text

\usepackage{hyperref}
\hypersetup{
	colorlinks=true, % Whether to color the text of links
	urlcolor=black, % Color for \url and \href links
	linkcolor=black, % Color for \nameref links
	citecolor=black, % Color of reference citations
}

 
\usepackage{titlesec} % Required for modifying sections
\titleformat{\section}[block]{\normalsize\bfseries}{\thesection .}{0.5em}{}[]
\titlespacing*{\section}{0pt}{0.25in}{0.125in} % Spacing around section titles, the order is: left, before and after
% \titlespacing*{\maketitle}{2pt}{8pt}{4pt}
% \renewcommand{\familydefault}{\rmdefault} % no effect
% \the\fontdimen2\font
% \fontdimen2\font=3.4pt

\title{
    \vspace{-40.5px}
    \Large 
        Effects of Operating Conditions on the Gas-Phase Decomposition of \\
        MTS/H$_2$ during Chemical Vapor Infiltration of SiC
    }
\author{
    {\normalsize
        R. Ranjan\textsuperscript{a,}\thanks{\footnotesize Corresponding author. \\ 
        \hspace*{12.5pt} \emph{Email address:} \textbf{\href{mailto:reetesh-ranjan@utc.edu}{reetesh-ranjan@utc.edu}} (R. Ranjan ) \\
        \hspace*{13pt} Notice: This manuscript has been authored by UT-Battelle, LLC, under contract DEAC05-00OR22725 with the US Depart-ment of Energy (DOE). The US government retains and the publisher, by accepting the article for publication, acknowledges that the US government retains a nonexclusive, paid-up, irrevocable, worldwide license to publish or reproduce the published form of this manuscript, or allow others to do so, for US government purposes. DOE will provide public access to these results of federally sponsored research in accordance with the DOE Public Access Plan (\bf \href{http://energy.gov/downloads/doe-public-access-plan}{http://energy.gov/downloads/doe-public-access-plan}).
} ,
        J. Wilson\textsuperscript{a},
        M. Barisik\textsuperscript{a},
        V. Ramanuj\textsuperscript{b},
        R. Sankaran\textsuperscript{b}
        }
    }
\date{
    \footnotesize 
        \emph{
            \textsuperscript{a}Department of Mechanical Engineering, The University of Tennessee Chattanooga \\
            615 McCallie Avenue, Chattanooga, TN 37403, USA \\
            \textsuperscript{b}Computational Sciences and Engineering Division, Oak Ridge National Laboratory \\
            1 Bethel Valley Rd., Oak Ridge, TN 37831, USA \\
        }
    }

\begin{document}
\maketitle
\noindent\hrulefill%
\vspace{-0.125in}
\section*{Abstract}
The quality of the silicon carbide (SiC) matrix composite, which has excellent thermo-mechanical properties,
fabricated by the well-established chemical vapor infiltration (CVI) process, depends upon the gas-phase
decomposition of the employed precursor and the subsequent heterogeneous reactions at the deposition
surface. The decomposition process is primarily affected by the operating parameters such as temperature,
pressure, gradient of pressure and temperature, and the composition of the incoming gas-phase reactants. In
this study, we perform plug flow reactor (PFR) analysis to examine the effects of operating parameters on the
decomposition of the mixture of methyltrichlorosilane (MTS) and hydrogen (H2 ), a widely used precursor
for the manufacturing of SiC matrix composite using the CVI process. The PFR analysis is performed using
three chemical mechanisms of increasing degree of complexity, which include globally reduced (1 step and 5
species), moderately complex (30 steps and 20 species), and detailed (103 steps and 42 species) mechanisms.
First, we discuss the reduction strategy to obtain the moderately complex mechanism from the detailed
one and assess the performance of the three mechanisms by comparing PFR results at atmospheric pressure
with reference experimental and numerical results. Afterward, the PFR analysis is performed using these
mechanisms at conditions relevant to CVI. These include a range of temperature (1100 K to 1600 K), pressure
(5 Torr to 100 Torr), temperature gradient (-190 K/m to -19 K/m), and pressure gradient (-7.7 Torr/m to
-0.97 Torr/m). The analysis of the results pertaining to the decomposition of MTS and production of several
intermediate reactive species shows that the decomposition of MTS is more sensitive to temperature and
its gradient than pressure or pressure gradient. The effects of the ratio of MTS and H2 in the incoming
mixture (0.05 to 0.2) are examined by considering isothermal and temperature-gradient conditions. The
results show enhanced decomposition of MTS in the case with the presence of a temperature gradient for
the same mixture composition, and a nonlinear dependence of the decomposition of MTS and production
of intermediates and byproducts on the incoming mixture composition. The present study also showed that
the moderately complex chemical kinetics show good agreement with the detailed mechanism, whereas the
globally reduced chemical mechanism yielded inaccurate results, thus implying that the moderately complex
mechanism can be used while performing detailed three-dimensional modeling of the CVI process.
\vspace{6pt}
\\
\emph{Keywords:} Chemical vapor infiltration, silicon carbide, methyltrichlorosilane, plug flow reactor, chemical mechanisms

\noindent\hrulefill

\section{Introduction}
Materials that can withstand high-temperature environments are required in many engineering applica-
tions such as nuclear reactors, automotive, spacecraft, aircraft components, furnace linings, power electronics,
\pagebreak

\noindent and cutting and grinding tools. Such materials include metals and alloys (titanium alloys, tungsten, molyb-
denum), ceramics (alumina, silicon carbide (SiC), graphite), and composites (ceramic matrix composites,
carbon-carbon composites) [1–5]. SiC is one such material, which has excellent properties such as high
strength, high thermal conductivity, low thermal expansion, chemical inertness, and good thermal shock
strength [6–9]. However, it also tends to have a low toughness [10–12], which has motivated interest in high
purity and high quality SiC fiber-reinforced matrix composites [13–19]. In such composites, a SiC matrix is
deposited within a fiber preform composed of high-purity, near-stoichiometric SiC fiber. Such an approach
leads to a better fracture toughness property while still having a similar performance under high thermal-
loading conditions. The deposition of the SiC matrix composites depends upon the operating conditions,
which affect the gas-phase decomposition and the subsequent heterogeneous surface reactions on the fiber
preform. The present study examines the effects of operating conditions on the gas-phase decomposition of
the precursor and the production of intermediate reactive species during SiC matrix deposition.

SiC matrix composites can be fabricated using approaches such as melt infiltration [20, 21], polymer
infiltration and pyrolysis [22, 23], and chemical vapor infiltration (CVI) [24–26]. Past studies have shown
that CVI is a reliable approach to obtain a very high-purity SiC matrix composite [25, 27]. During the CVI
process, a SiC precursor is introduced into a chamber in the gas phase, which diffuses into the fiber preform
and chemically reacts, forming a SiC matrix within the sample. There are two major challenges associated
with the CVI process, which affect the overall quality of the resultant matrix composite. The first challenge
is associated with the processing time. While CVI produces a very high-purity matrix, as the deposition
process is dependent on the diffusion of reactants into the fiber preform, ideally, a slow reaction rate is
desirable to ensure uniform transport of reactants throughout the preform. However, with a slower reaction
rate, the processing time gets longer, which can affect the quality of the resulting composite. For example,
high porosity can be observed, and filling of inter-tow sharp and large pores can be a challenge while keeping
the processing times low [28]. Therefore, modifications have been considered to the conventional CVI process
[25, 29]. These include thermal-gradient CVI [30], pressure-gradient CVI (forced flow CVI) [26], pressure-
pulsed CVI [31], and whisker-growing assisted CVI [32]. The second challenge is related to the dependence
of the quality of the SiC deposition on the gas-phase decomposition of the precursor that is used during the
process. Note that the precursor decomposition and production of the reactive intermediates directly affect
the heterogeneous surface reactions and can also affect the overall processing time. This particular challenge
can be addressed by studying the effects of operating conditions on the decomposition of the precursor used
for the CVI of SiC matrix composite, which is the major focus of this study.

Some of the commonly used precursors for the fabrication of SiC through the CVI process include
methyltrichlorosilane (CH3SiCl3 ), hereafter referred to as MTS, silane (SiH4 ) along with hydrocarbons such
as methane (CH4 ), dichloromethylsilane (CH3 SiHCl2 ), and hexamethyldisilane ((CH3 )6 Si2) [33–35]. Out
of these, MTS is widely used as it yields a perfect stoichiometry (1:1 Si:C), leading to high-purity SiC
with minimal free carbon or silicon [25]. In addition, its decomposition characteristics can be controlled by
varying operating temperature (800 to 1000 o C) and pressure (5 to 100 Torr) conditions, and the ratio of
MTS to hydrogen (H2 ) in the incoming mixture. Note that MTS is used as a precursor along with H2 as
the carrier gas, which in turn can suppress free carbon formation and reduce volatile intermediates such as
SiCl4 [35]. Therefore, in this study, we study the decomposition characteristics of MTS, which can lead to
the production of several intermediate reactive species such as CH4 , HCl, SiCl4, SiHCl3 , etc., that in turn
affect the deposition process at the preform surface where heterogeneous reactions occur.

The computational study of CVI of SiC matrix composite can be carried out using techniques with
varying levels of fidelity and computational cost, which include, computational fluid dynamics (CFD) [36–
39], direct simulation Monte Carlo (DSMC) [26, 40], or plug-flow reactor (PFR) [41]. For example, while
CFD can simulate large-scale fabrication systems and account for transport and diffusion processes, it faces
challenges in accurate modeling of the matrix infiltration process, where the transport of the reactants
occurs through the fiber preform, leading to Knudsen numbers much larger than unity. To this end, the
DSMC technique can be employed, although it is computationally expensive for the simulation of large-scale
fabrication systems. While using either of these techniques, accurate modeling of chemical kinetics is pivotal
to the accuracy of the prediction of the gas-phase decomposition and subsequent deposition through the
heterogeneous surface reactions on the fiber preform. However, due to computational complexities associated
with detailed chemical kinetics, often simplified chemical mechanisms are employed while using CFD and
DSMC techniques [40, 42, 43], which tend to yield inaccurate results. Therefore, there is a need to have
reliable yet affordable chemical mechanisms that can accurately predict the gas-phase decomposition and
subsequent heterogeneous reactions during CVI of SiC. In this study, we employ a plug-flow reactor (PFR)
% insert figure 1
\\
as the computational technique, which can efficiently account for detailed chemical kinetics [44–47] to study
the decomposition of MTS under different operating conditions. In a PFR, a steady flow field within an axial
duct is modeled, where the flow in the transverse direction is considered to be completely homogeneous.

The key objective of this study is to first establish a PFR setup, and then to use the setup to examine
the effects of operating conditions on the gas-phase decomposition of MTS and production of intermediate
reactive species such as CH4 , HCl, SiHCl3 , SiCl2 , and SiCl4 under CVI-relevant conditions. Past studies
using PFR have primarily focused on the development and assessment of chemical kinetics, and therefore,
the present study will complement the findings of such studies. Along with the detailed kinetics, we also
assess the performance of two other chemical mechanisms, which can be potentially used with CFD and
DSMC simulations. The three chemical mechanisms include detailed (103 steps and 43 species) [48–50],
moderately complex reduced (30 steps and 20 species), and globally reduced (1 step and 5 species) [51]
mechanisms. The moderately complex mechanism is obtained using a reduction technique (discussed later)
in the present study. To examine the decomposition of MTS in the presence of H2 , we consider different
CVI reactor conditions. Apart from considering a range of operating temperatures (1100 K to 1600 K) and
pressures (5 Torr to 100 Torr), we also examine the effects of a range of temperature gradients (-190 K/m
to -19 K/m), and pressure gradients (-7.7 Torr/m to -0.97 Torr/m). Past studies have also shown that the
decomposition of MTS is affected by the composition, i.e., molar ratio of MTS/H2 ($\beta$) [52]. Therefore, the
effects of variation in $\beta$ (0.05 to 0.2) are also examined after identifying optimal operating temperature and
pressure conditions.

This article is arranged as follows. The details for the computational setup and the employed numerical
methodology are presented in Section 2. In Section 3, an assessment of the PFR-based computational
strategy is performed by comparing the results with reference data from the literature. The results of this
study are discussed in Section 4. Finally, the key outcomes of this study are summarized in Section 5.
\section{Setup and Approach}
In this section, we rs t describ e the details of the CVI reactor c onguration. Afterward, the computational
metho d is des crib ed. Finally, the details of the chemical m echanisms used in this study are presented.
2.1. Details of the Setup
We consider a cylindrical hot-wall 
ow CVI reactor conguration. A schematic of this reactor is shown in
Fig. 1. The dimensions of the reactor follow a past study [41]. The axial extent of the cylindrical reactor is
L
= 0
:
52 m, and the inner radius is
R
= 0
:
0048 m. At the inlet of the reactor, a premixed mixture of MTS,
H
2
, and Ar ente rs at pressure
P
and temp e rature
T
. The molar comp osition at the inlet of the reactor is
95\% Ar and 5\% mixture of MTS and H
2
. The ratio of molar comp os ition of MTS and H
2
is sp ecied to b e

, which is varied in this s tudy from 0.05 to 0.2.
3


2.2. Computational Approach
The chemically reacting 
ow in the CVI reac to r is simulated using a laminar PFR mo del. Such an
approach has b een followed in past studies for mo deling the 
ow within CVD and CVI reactors [41, 45, 46 ].
A schematic of PFR is shown in Fig. 1. In a PFR, the ste ady laminar 
ow within a duct is mo deled,
where the gase ous 
ow in the transverse direction to the axial (
x
) direction is considered to b e completely
homogeneous. Therefore, changes in the state of the gas can only o ccur along the
x
direction. Furthermore,
in this mo deling approach, all diusion pro cesses are neglected. As stated in 1, compared to a conventional
CFD, even though a PFR is a much simpler represe ntation of the 
ow, it can include detailed chemical
kinetics, thus providing a computationally effcient approach for the simulation of physical phenomena such
as pyrolys is, catalytic pro cesses, and emissions.
A PFR is characterized by the state variables, which inc lude de ns ity (
ˆ
), temp erature (
T
), pressure (
P
),
mass fraction of
k
th
sp ecies, (
Y
k
) and the surface coverage of the
k
th
sp ecies (
S
k
). Here, we consider the
net pro duction of surface sp ecies to b e zero, which implies the total surface coverage to b e 1. T he governing
one-dimensional (1D) equations for PFR include conservation of mass, momentum, energy, and sp ecies mass
equations, which are given by
u
dˆ
dx
+
ˆ
du
dx
= 0
;
(2.1)
ˆu
du
dx
=

dP
dx
;
(2.2)
ˆuc
p
dT
dx
=

N
X
k
=1
h
k
\_
!
k
;
(2.3)
ˆu
d Y
k
dx
= \_
!
k
W
k
;
for
k
= 1
;
2
; : : : ; N :
(2.4)
Here,
c
p
denotes the sp ecic heat of the mixture at constant pressure,
h
k
and \_
!
k
denote the molar enthalpy
and reaction rate of the
k
th
sp ecies, resp ectively, and
N
denotes the total numb er of sp ecies. These equations
are further supplemented by the ide al gas equation of state

ˆ
=
P
W
R
u
T

for relating the thermo dynamic
prop erties, and nite-rate chemical kinetics for determining \_
!
k
. Here,
W
and
R
u
denote the mixture molec-
ular weight and universal gas constant, res p ectively. The system of governing equations is complete by
sp ecifying the inlet conditions. Note that, as the diusion pro cess is neglected, downstream parts of the
reactor have no in
uence on upstream parts. Therefore, PFR can b e integrated as initial value problems,
starting from the comp osition at the inle t and moving towards the outlet by considering nite size volume
(
V
), referred to as plugs, as shown in Fig. 1.
In the present s tudy, the reactor length that is b eing considered here has a residence time of ab out 30
ms, which implies a negligible radial diusion/stratication [41]. Therefore, the PFR mo del utilizes a longer
residence time of 500 ms than this critical limit. The PFR simulations are carried out using the Cantera
software [53], which is a well-established op en-source software for the simulation of proble ms p ertaining to
thermo dynamics, chemical kinetics, and transp ort pro cesses.
2.3. Chemical Kinetics Mechanisms
We consider three chemical mechanisms with dierent levels of delity. These include a detailed 42
sp ecies and 103 ste ps mechanism [48-50], a mo derately complex 20 sp ecies and 30 steps mechanism, and a
globally reduced 5 sp ecies and 1 step mechanism [51]. Hereafter, these mechanisms are referred to as M1, M2,
and M3, resp ectively. Here, the mechanism M1 se rves as a reference for comparing the p erformance of the
other two mechanisms. The globally reduced mechanis m M3, as rep orted in past studies [41, 51], tends to b e
inadequate to describ e the decomp os ition of MTS due to very few chemical sp ecie s. The mo derately complex
mechanism M2 can b e p otentially useful for CFD and DSMC s imulations, as it employs 20 sp ecies, which
can p otentially capture the ee cts of kinetics while still b eing cost-eective compared to the M1 mechanism.
While M1 and M3 mechanisms are obtaine d from past studies, M2 is obtained by p erforming a reduction of
the M1 mechanism using the Cantera software [53]. The reduction strategy simulate s an adiabatic constant
pressure rea ctor over a range of pressure and temp erature conditions relevant to this study, tracks the
maximum reaction rates for each re action, and identies the imp ortant reactions based on the relative net
reaction rate. This is followed by creating a series of reduced chemical mechanisms, including only the top
4


Table 1: List of sp ecies in the M1, M2, and M3 che mical mechanisms.
Mechanism
Sp ecies
M1
CH
3
SiCl
3
, H
2
, CH
3
, SiCl
3
, H, CH
2
SiCl
3
, Cl, CH
3
SiCl
2
, CH
2
SiCl
2
, HCl, SiCl
2
, CH
2
,
SiHCl
3
, CHSiCl
3
, CH
3
SiCl, Cl
2
, CH
4
, C
2
H
5
, CH
2
Cl, CH
2
Cl
2
, C
2
H
4
, C
2
H
2
, C
2
H
3
,
CH
2
C, CH
3
C, C
2
H, C
2
H
5
Cl, C
2
H
3
Cl, SiCl
4
, Si
2
Cl
5
, Si
2
Cl
4
, Si
2
Cl
6
, SiHCl
2
, SiH
2
Cl
2
,
SiHCl, SiH
2
Cl, SiH
3
Cl, CH
3
SiHCl
2
, CH
2
SiHCl, C
2
H
6
, CH
3
SiH
2
Cl, Ar
M2
SiCl
4
, CH
2
SiCl
2
, H, H
2
, SiHCl
3
, SiCl
2
, SiHCl
2
, SiCl
3
, HCl, SiH
3
Cl,
CH
2
SiCl
3
, CH
4
, SiH
2
Cl
2
, CH
3
SiHCl
2
, CH
3
SiCl
2
, CH
3
, Cl, SiHCl, CH
3
SiCl
3
, Ar
M3
CH
3
SiCl
3
, H
2
, CH
4
, SiHCl
3
, Ar
reactions and the as so ciated sp ecies, and p erforming the simulations again with these me chanisms to see
whether the reduced mechanisms with a certain numb er of sp ecies are able to adequately simulate the reac tor.
The comparison of results from dierent reduced chemical mechanisms with the detailed M1 mechanism and
dominant reac tions obtained for ma jor sp ecies using sensitivity analysis is discussed in Sec. 3.1.
\section{Assessment of Computational Strategy}
In this section, we rst discuss the results from the reduction of the M1 mechanism yielding the M2
mechanism. Afterward, we compare the computational cost of the three mechanisms. Finally, we p erform
a verication and validation study using these mechanisms to establish the PFR-based strategy for further
studies.
3.1. Reduction of Chemical Mechanism
To p erform the reduction of the detailed mechanism M1, we simulate adiabatic constant-pressure reactors
for 200 ms. We examine the accuracy of reduced mechanisms with increasing de gre e of complexity for a
range of reactor temp eratures (1100 K to 1600 K) and pressures (5 Torr to 100 Torr). The numb er of
chemical reactions to b e used in the reduced mechanism was varied from 10 to 50. As discussed in Sec. 2.3,
the reduction strategy identie s the most active reac tions . It re sulted in mechanisms with the numb er of
sp ecies varying from 12-28, with 10-50 reactions.
Figures 2 and 3 show the variation of mole fraction of MTS (
X
MTS
) and CH
4
(
X
CH
4
), resp ectively
with resp ect to time. These sp ecies are chos en as they tend to b e dominant sp e cies that participate in
the homogeneous reactions and can further aect the heterogeneous surface reactions. At T = 1100 K, all
reduced mechanisms with 10-20 reactions yield excellent agreement with the re fe re nc e M1 m echanism at
low (5 Torr) and high (100 Torr) pressure. The dierences tend to o ccur only at highe r temp eratures (T =
1300 K and T = 1600 K), particularly with the reduced mechanism having 10 reactions. Such a die re nc e
with 10 reactions is observe d for b oth MTS decomp osition and CH
4
pro duction. This can b e attributed to
enhanced decomp osition/pro duction of MTS/CH
4
at higher temp erature, which is aected by the presence of
reactions involving reactive interme diates when 20 or more reactions are included. At 1300 K, the pressure
sensitivity is also observed as the error magnitude in the variation of
X
MTS
and
X
CH
4
with the 10-step
mechanism is increased at higher pressure. At the highest temp erature (
T
= 1600 K) and highest pre ssure
(
P
= 100 Torr), error is also observed with 20-step mechanisms. Based on these res ults , it can b e inferred
that a mechanism with 20 sp ecies and 30 reactions yields accurate results for the range of te mp eratures and
pressures considered here. Therefore, for further studies, we consider the M2 mechanism to have 20 sp ecies
and 30 reactions. Table 1 lists all the sp ecies included in the three mechanisms.
To further examine the dominant reactions corresp onding to dierent sp ecies while using M1 and M2
mechanisms, a sensitivity analysis for dierent sp ecies is p erformed. The results from this analysis are shown
in Fig. 4 for MTS, CH
4
, and HCl. With b oth M1 and M2 mechanis ms, the three dominant reactions are the
same, which are given by
CH
3
SiCl
3
\$
CH
2
SiCl
2
+ HCl
;
(R1)
CH
3
SiCl
3
\$
CH
3
+ SiCl
3
;
(R2)
CH
3
+ H
2
\$
CH
4
+ H
:
(R3)
Out of these three reactions, MTS is more sensitive to R1 and R2. Thes e reactions play a critical role in the
decomp osition of MTS and leads to formation of reactive intermediates such as CH
2
SiCl
2
, CH
3
, and SiCl
3
,
5


(a)
P
= 5 Torr,
T
= 1100 K
(b)
P
= 100 Torr,
T
= 1100 K
(c)
P
= 5 Torr,
T
= 1300 K
(d)
P
= 100 Torr,
T
= 1300 K
(e)
P
= 5 Torr,
T
= 1600 K
(f )
P
= 100 Torr,
T
= 1600 K
Figure 2: Comparison of time evoluti on of the mole fraction of MTS obtained using chemical mechanisms with dierent levels
of complexity at T = 1100 K (a, b), T = 1300 K (c , d), and T = 1600 K (e, f ) at P = 5 Torr (a, c, e) and P = 100 Torr (b, d,
f ). Here, K and R denotes the numb er of sp ecies and chemi cal reactions, resp ecti vely.
6


(a)
P
= 5 Torr,
T
= 1100 K
(b)
P
= 100 Torr,
T
= 1100 K
(c)
P
= 5 Torr,
T
= 1300 K
(d)
P
= 100 torr,
T
= 1300 K
(e)
P
= 5 Torr,
T
= 1600 K
(f )
P
= 100 Torr,
T
= 1600 K
Figure 3: Comparison of time evolution of the mole fraction of CH
4
obtained using chemical mechanisms with dierent levels
of complexity at T = 1100 K (a, b), T = 1300 K (c , d), and T = 1600 K (e, f ) at P = 5 Torr (a, c, e) and P = 100 Torr (b, d,
f ). Here, K and R denotes the numb er of sp ecies and chemi cal reactions, resp ecti vely.
7


(a) MTS (M1)
(b) MTS (M2)
(c) CH
4
(M1)
(d) CH
4
(M2)
(e) HCl (M1)
(f ) H Cl (M2)
Figure 4: Six dominant reacti ons for MTS (a, b), CH
4
(c, d), and HCl (e, f ) obtai ned using sensitivity analysis with M1 and
M2 mechanisms.
8


Table 2: Comparison of computational cost for 1 PFR simulation using M1, M2, and M3 mechanisms.
Mechanism
Computational Cost [s]
Relative Cost [\%]
M1
10.4
100
M2
4.6
44.2
M3
2.7
26
which can further decom p ose, combine, or participate in heterogeneous surface reactions. Additionally,
R1 also leads to HCl, which is a bypro duct resp onsible for gas-phase etching [54, 55]. The R3 reaction is
primarily resp onsible for the formation of CH
4
, which can help avoid c ontamination, and the radical H,
which is crucial for the removal of Cl and the promotion of SiC growth through heterogeneous reactions [56].
For CH
4
, R1 and R3 are the rst two dominant reactions (see Fig. 4(c) and (d)), which are found in
b oth M1 and M2 mechanisms. However, the third dominant reaction is dierent for CH
4
with M1 and M2
mechanisms. While the M1 mechanism shows the third dominant reaction to b e decomp osition of C
2
H
6
into
CH
3
radical through the reaction
C
2
H
6
\$
2CH
3
;
the M2 mechanism shows the third dominant reaction as:
CH
2
SiCl
3
+ H2
\$
CH
3
SiCl
3
+ H
:
This dierence could b e due to the absence of C
2
H
6
in the M2 mechanism (see Table 1). Although the third
dominant reaction for CH
4
diers for the M1 and M2 mechanisms, the s ensitivity of the rst two dominant
reactions is much higher compared to the third reaction.
An imp ortant bypro duct during the de comp osition of MTS is HCl. With b oth M1 and M2 mechanisms,
similar to MTS, R1 and R2 are the rs t two dominant reactions (s ee Fig. 4(e) and (f )), whereas the third
dominant reac tion is
SiCl
3
\$
SiCl
2
+ Cl
:
This is an intermediate and critical reac tion, as SiCl
2
is a direct precursor to SiC formation via heterogeneous
reaction with H
2
or CH
2
at the surface [44]. On the other hand, the radic al C l c an either react with H
2
,
leading to the formation of HCl, which in turn can prevent the formation of the undesirable Cl-rich bypro ducts
such as SiCl
4
.
To summarize, the results discuss ed in this section from the adiabatic reactor simulations and sensitivity
analysis demonstrate that the reduced M2 mechanism can b e considered adequate to capture the ke y asp ects
of MTS decomp osition in the presence of H
2
in comparison to the detailed M1 mechanism for the conditions
relevant to this study.
3.2. Computational Cost
As the numb er of sp ecies and reactions tends to dier in the three chemical mechanisms , it is exp ected
that the computational cos t will dier. A key fo cus of the present work is to demonstrate that mo derately
complex chemical kinetics, s uch as the M2 m echanism, can yield accurate results, which in turn can b e used
with CFD simulations. Here, we assess the computational cost of the three mechanisms by simulating a
PFR at a pressure of 5 Torr and a temp erature of 1350 K. The computational cost comparison is shown in
Table 2.
As exp ected, the computational cost of the M2 and M3 mechanis ms is lower than the detailed M1
mechanism. The cost of M2 decreases primarily due to a decrease in the numb er of sp ecies compared to M1,
from 42 to 20, with a much less impact of a decrease in the numb er of reactions. Note that mechanisms such
as M2 can b e used for CFD simulations, as mechanisms with 15-20 chemical sp ecies tend to b e manageable
while employing nite-rate kinetics. However, it should b e noted that the cost of a CFD simulation, apart
from including the kinetics cost, also includes the cost of computation of convective and diusive 
uxes,
which can b e comparable to or larger than the cost of kinetics compared to a PFR simulation. The cost
of M3 do es not linearly scale with the numb er of sp ec ie s, which c an b e asso c iated with a reduc ed rate of
convergenc e of the PFR simulation with the M1 mechanis m.
9


(a) MTS
(b) CH
4
Figure 5: Vari ati on of mole fraction of MTS (a) and CH
4
(b) with resp ect to temp erature. Exp erimental data and `Ref` data
corresp onds to exp erimental and computational study using detailed mechanism [41].
3.3. Verication and Validation
We p erform the verication and validation of the PFR-based strategy by simulating MTS decomp osition
in the pre sence of H
2
at an op erating pressure of 1 atm (760 Torr), inlet c omp osition ratio of 0.1, and over a
range of temp eratures varying from 1100 K to 1600 K. The results are compared in Fig. 5, w hich includes the
variation of the mole fraction of MTS and CH
4
with resp ec t to temp erature. For comparison, we also include
exp erimental and computational results obtained using a detailed chemical mechanis m from a past study
[41]. Note that the detailed mechanism us ed in the reference s tudy was devised to match the exp erimental
data. Furthermore, the op erating pressure of the present conguration is much higher than that of a typical
CVI, which is usually op erated at lower pressures.
Both M1 and M2 mechanisms yield the same results for MTS, which starts to decomp ose gradually
from 1100 K to 1200 K, and then rapidly leads to a ne arly complete decomp osition by 1350 K. The value s
of MTS by these mechanis ms are over-predicted compared to the reference results. Although the rate of
decomp osition of MTS from the M1 and M3 me chanisms matches during early and later stages with the
reference com putational results, the decom p osition of MTS is delayed with resp ect to temp erature. The
results from the M3 mechanism are under-predicted compared to the reference results and s how an e arly
and faster rate of decomp os ition of MTS. Typically, one-step mechanisms, s uch as the M3 me chanism, are
devised so that the variation of ma jor sp ecies is captured well, which is not the case here, thus indicating its
inaccuracy.
The results for the variation of CH
4
mole fraction (see Fig. 5(b)) from the M1 and M2 mechanisms tend to
match till 1350 K. At temp eratures b eyond 1350 K, the M1 mechanism show reduced levels of CH
4
implying
its further decomp osition, whereas the M2 mechanism yields constant values of CH
4
. The dierences can
b e attributed to the eects of reduced numb er of sp ec ie s and reactions in the M2 mechanism compared
to the M1 mechanism, which can aect the the rm o dynamics and reaction pathways. However, for typical
CVI conditions of 800 to 1000
o
C, b oth mechanisms yie ld similar results, implying M2 can b e considered
to b e accurate with resp ect to M1. As Fig. 5(a) showed a delayed decomp osition of MTS with resp ect to
temp erature using M1 and M2 mechanisms, this le ads to a delayed pro duction of CH
4
compared to the
reference results. The p eak value of CH
4
and its variation at higher temp e ratures by the M1 mechanism
show go o d agreement with the reference computational results. Compared to exp erimental data, the p eak
values of CH
4
are ove r- predicte d by all mechanisms. With the M3 mechanism, the pro duction of CH
4
is
over-predicted as it co rrelate s with the under-prediction of MTS in Fig. 5(a).
The re sults shown here demonstrate that the PFR strategy employed here can capture the decomp osition
of MTS in the presence of H
2
. There are dierences in the results from the detailed M1 me chanism and
the detailed mechanism prop osed in the reference study [41], which was devised to capture the exp erimental
results. However, these dierences are asso ciated with a delayed start of decomp osition of MTS and pro-
duction of CH
4
with resp ect to temp erature, although the rates of de comp osition of MTS and pro duction of
CH
4
tend to match. In this study, we have not made any attempts to mo dify the baseline M1 mechanism, as
the e xp erimental data is only available at 760 Torr, where as this study examines CVI reactor congurations
at lower pressures. Therefore, the results from the M1 mechanism are considered as a reference in the rest
10


of this study to examine the eects of op erating conditions on the MTS decomp osition and to as sess the
p erformance of the M2 and M3 mechanisms.
4. Results and Discussion
In this section, we discuss the results obtained from the PFR simulations of the MTS/H
2
decomp osition
under a wide range of CVI-releva nt op erating conditions. First, we analyze the eects of op erating temp er-
ature at dierent values of pressure. Afterward, we examine the eects of op erating pressure at dierent
values of tem p erature. This is followed by the analysis of results from the cases em ploying temp erature and
pressure gradients at dierent values of pressures and tem p eratures. Finally, the eects of the molar ratio
of MTS to H
2
(

) at the inlet of the CVI reactor are examined under two reactor congurations, which
include isothermal and temp erature-gradient conditions across the reactor for optimal op erating pressure
and temp erature conditions.
In a CVI relevant conditions, the decomp osition of MTS/H
2
leads to the formation of se veral intermediate
reactive sp ecies. We examine the concentration of several sp ecies, such as MTS, CH
4
, SiHCl
3
, SiCl
4
, HCl,
and SiC l
2
, some of which also participate in the heterogeneous s urface reactions. At high temp eratures,
CH
4
pro duced during the decom p osition of MTS can further decomp ose through thermal cracking providing
a sourc e of carb on [57], which dep ending up on the op erating conditions can contribute to SiC formation
[58], or can lead to unwanted residual carb on [59, 60]. Furthermore, it acts as a diluent to the reactive gas
phase, as it can also comp e te with MTS. The hydrogenation of the Si-C b ond in MTS and substitution of
methyl (CH
3
) radical with H
2
leads to the form atio n of SiHCl
3
. It is a reactive intermediate that can further
decomp ose, providing the Si ne eded for the SiC dep osition. It can also react further, leading to HCl, and
can aect the dep osition rate. However, if SiHCl
3
do es not react further, it can carry silicon away from the
dep osition zone, which can lower the SiC yield or lead to a Si-decient or C- rich matrix. SiCl
4
is also a
dominant and stable sp ecies, which is generally inactive toward SiC formation under typical CVI conditions,
however, it can act as a source of the Si [41, 61]. It can also aect the gas- phase equilibrium, overall Si/C
balance, and dep osition kinetics. SiCl
2
is a highly re active intermediate, which s erves as a key Si-b earing
sp ecies that readily reacts with c arb on-containing gases such as CH
4
, C
2
H
2
, or CH
2
to form solid SiC. For
example, the following heterogeneous reaction leads to the formation of SiC [44]:
SiCl
2
(g) + CH
2
(g)
!
SiC(s) + 2HCl(g)
:
Due to its high reactivity, SiCl
2
promotes rapid SiC growth near the inltration front, enhancing dep osition
effciency. However, excessive SiCl
2
may lead to non-uniform dep osition or p ore closure. Finally, HCl is
pro duced as a ma jor gaseous sp ecies, which, although it do es not participate directly in SiC formation,
can p otentially shift the chemical equilibria, suppressing undesirable gas-phase nucle atio n and enhancing
dep osition selectivity on solid surfaces.
It is evident that the aforementioned sp ecies play a crucial role in aecting the Si:C ratio, shifting of the
gas-phase equilibrium, and the dep osition kinetics, which in turn aect the SiC dep osition while including
the heterogene ous surface reactions. Therefore, we examine the concentration of all thes e sp ecies in the
subsequent discussion of results. Here onwards, the molar comp osition of the sp ec ie s is obtained at the exit
of the reactor and each PFR s imulation, as discussed b e fore in Sec. 2.2, is carried out for a residence time
of 500 ms within the reactor.
4.1. Eects of Temperature
To examine the eects of op erating temp erature (
T
), we simulate the PFR over a range of values of
T
from 1100 K to 1600 K. Additionally, each PFR s imulation is p erformed at three values of op erating pressure
(
P
), which include 5 Torr, 52.5 Torr, and 100 Torr. From these simulations, the mole fraction of several of
the sp ecies is obtained at the exit of the reactor, which are shown in Fig. 6.
It is evident from Fig. 6(a) that MTS tends to decomp ose at a slower rate for
T
varying from 1100 to
ab out 1180 K with b oth M1 and M2 mechanisms. However, b eyond this value of
T
, it tends to decomp ose
at a highe r rate. For
T >
1350 K, no MTS is observed at the exit of the PFR, implying its complete
decomp osition. These re sults are consistent with the past studies conducted at dierent pressure values
[41, 52, 62], which have also shown that MTS in the presence of H
2
starts to dec omp ose around
T
ˇ
1100
K and the rate of decomp osition increases with an inc re ase in
T
. Both M1 and M2 m echanisms yield
identical results, however, the globally reduced M3 mechanis m leads to a signicant over-prediction of MTS.
11


(a) MTS
(b) CH
4
(c) SiHCl
3
(d) SiCl
4
(e) HCl
(f ) Si Cl
2
Figure 6: Variation of mole fraction of MTS (a), CH
4
(b), SiHCl
3
(c), SiCl
4
(d), HCl (e), and SiCl
2
(f ) with resp ect to
temp erature at three values of op erating pressure (5 Torr, 52.5 Torr, and 100 Torr).
12


Furthermore, with M1 and M2 mechanisms, we do not observe any sensitivity of the results to
P
. However,
the sensitivity to
P
is evident in the results using the M3 mechanism, where in the case with
P
= 5 Torr,
the decomp osition of MTS tends to b e incomplete even at
T
= 1600 K.
The variation of
X
CH
4
with resp ect to
T
in Fig. 6(b) shows that it tends to increase with
T
, reaching a
p eak value at around 1350 K. The p eak value shows a minor decrease with an increase in
P
. This b ehavior is
captured by b oth mechanisms M1 and M2. Howe ver, a further increase in the value of
T
shows dierences in
the results from the M1 and M2 mechanisms. While a de crease in
X
CH
4
with resp ect to
T
o ccurs with the M1
mechanism, the M2 mechanism yields nearly constant values. The rate of decrease of
X
CH
4
with mechanism
M1 increases with an increase in
P
. As discussed b efore, C H
4
can play a critical role in the SiC dep osition
pro cess, as its further decomp osition can provide a source of carb on, which, dep ending up on c onditions,
can c ontribute to SiC formation or can lead to unwanted residual carb on. At around
T
= 1200
to
1300 K,
X
MTS
=X
CH
4
/
1, which is considered suitable for promoting stoichiometric SiC formation. From these
results, we can also infer that M2 can b e considered to yield reasonable agreement with M1 till
T
ˇ
1400 K.
The results from the globally reduced M3 mechanism dier substantially from the other two mechanisms. As
MTS tends to decomp ose at higher values of
T
(see Fig. 6(a)), it aects the pro duc tion of CH
4
. Furthermore,
similar to the sensitivity of MTS decomp osition, the sensitivity of CH
4
pro duction/consumption to
P
is also
evident, particularly at the lower pressure.
The variation of
X
SiHCl
3
and
X
SiCl
4
shown in Figs. 6(c) and (d) with resp ect to
T
tends to b e similar
with mechanisms M1 and M2. It is observed that their values increase with
T
and attain a maximum around
T
= 1300 K, which is followed by a decrease with a further increase in
T
. T he sensitivity to
P
is clearly
evident, where we obse rve that the p eak value increases with
P
. The dierences b etween M1 and M2 tend
to o ccur in the variation of
X
SiCl
4
, where the p eak values by M2 are under-predicted. However, compared
to other chemical sp ecies, the quantitative value of
X
SiCl
4
tends to b e an order of magnitude smaller. With
the M3 mechanism, the p eak value of
X
SiHCl
3
is signicantly over-predicted compared to the other two
mechanisms. As discussed b efore, SiHCl
3
is a parasitic bypro duc t, which carries Si away from the dep os ition
zone, which can lower the SiC yield or lead to a Si-decient or C- rich matrix. Therefore, using the M3
mechanism can lead to inaccurate prediction of SiC dep osition.
Both HCl and SiCl
2
show a similar variation with
T
(see Figs. 6(e) and (f )), where we can observe an
increase with
T
till ab out 1350 K, which is followed by nearly constant value s with
T
. Furthermore, there
is no sensitivity to
P
and b oth mechanisms yield the sa me re sults , again demonstrating the accuracy of the
M2 mechanism. As discussed b efore, b oth HCl and SiCl
2
can b e advantageous for SiC dep os ition, however,
their excess comp osition should b e avoided. From the results, we can infer that
T <
1300 K leads to
X
HCl
and
X
SiCl
2
less than their corresp onding maximum values.
Based on the results discussed in this section for the eects of
T
, we can infer that 1250 K
/
T
/
1300 K
can b e considered a re asonable range for the op erating temp erature of the reactor. In this range of
T
, the
decomp osition of MTS is adequate, and the formation of bypro ducts is lower than their corresp onding p eak
values. Furthermore, the M2 m echanism tends to yield com parable results to those obtained us ing the M1
mechanism, while the M3 m echanism, as exp ected, tends to yield inaccurate res ults. The sensitivity to
P
is evident on several of the bypro ducts during the decomp osition of MTS, which is discussed further in the
next section.
4.2. Eects of Pressure
We exam ine the eects of
P
by varying it from 5 to 100 Torr at three dierent values of
T
. Based on
the res ults discussed in Sec. 4.1, we observe that the ma jor sp ecies showed signicant variation for
T
within
the range of 1150-1350 K, therefore, we consider three values of
T
, namely, 1150 K, 1250 K, and 1350 K.
Figure 7 shows the variation of mole fraction of s ix sp ecies with resp ect to
P
at the exit of the reactor.
We can observe in Figs . 7(a) and (b) that b oth MTS and CH
4
do not show much sensitivity to the variation
in
P
, particularly while using M1 and M2 mechanisms. Past studies unde r CVD conditions have also shown
weak sensitivity to
P
on the decomp osition under typic al pyrolysis conditions (800-1400
o
C). However, the
M3 mechanism do es exhibit the eects of pressure, which leads to an increase in decomp osition/pro duction
of MTS/CH
4
, particularly at higher temp eratures (1250 K and 1350 K). This again shows the inaccuracy of
the globally reduc ed mechanism.
The variation of other sp ecies such as SiHCl
3
and SiCl
4
(see Figs. 7(c) and (d)) do es exhibit sensitivity
to the varia tion of
P
with all three mechanisms. The mole fraction of these sp ecies te nds to increase with
P
at all three values of
T
, although the increase in
T
leads to an augmentation of the rate of increase of
pro duction of thes e sp ecies with
P
. Such an increas e in the pro duc tion of SiHCl
3
with an increase in
P
is due
13


(a) MTS
(b) CH
4
(c) SiHCl
3
(d) SiCl
4
(e) HCl
(f ) Si Cl
2
Figure 7: Variation of mole fraction of MTS (a), CH
4
(b), SiHCl
3
(c), SiCl
4
(d), HCl (e), and SiCl
2
(f ) with resp ect to pressure
at three values of op erating temp erature (1150 K, 1250 K, and 1350 K).
14


to its formation pathway b eing governed by gas-phase equilibrium reactions and radical-driven mechanis ms.
For example, SiHCl
3
is generated via the secondary gas-phase reactions involving MTS fragments and HCl
through the following reactions
CHSiCl
3
!
CH
3
+ SiCl
3
(Pyrolysis step)
;
SiCl
3
+ HCl
\$
SiHCl
3
+ Cl (Equilibrium limited step)
:
In the ab ove equations, the equilibrium is sensitive to
P
, as a higher value of
P
shifts the equilibrium toward
fewer gas-phase sp ecies and a lower value of
P
suppresses reverse reactions, reducing SiHCl
3
yield. Moreover,
the gas- phase radicals, such as SiC l
3
and CH
3
p ersist longer due to fe wer collisions at lower pressure (
P <
10
Torr), whereas at higher
P
, the increased collision frequency promotes radical recombination, enhancing the
pro duction of SiHCl
3
. For similar reas ons , the pro duction of SiCl
4
increases with
P
as its formation is also
governed by gas-phase equilibrium reactions and chlorine redistribution pathways. While the M2 mechanism
shows excellent agreement with the M1 mechanism for the pro duc tion of SiHCl
3
, it tends to underpredict
the pro duction of SiCl
4
, particularly at
T
= 1250 K and 1350 K. This discrepancy can b e attributed to the
absence of the following reaction from the M2 mechanism, which is another pathway of pro duction of SiCl
4
SiCl
3
+ HCl
\$
SiCl
4
+ H (Equilibrium limited step)
:
The M3 mechanism shows signicant overprediction of SiHCl
3
, which again underscores its inaccuracy.
Unlike SiHCl
3
and SiCl
4
, and similar to MTS and CH
4
, the mole fractions of HCl and SiCl
2
exhibit
insensitivity to
P
at the three values of
T
. The pro duction of HCl primarily dep ends up on temp erature
and mass 
ow rate/residence time. Similarly, the pro duction of SiCl
2
shows insensitivity to
P
as it is a
short-lived intermediate sp ecies, and its formation is governed by unimolecular decomp osition of MTS or its
fragments. For example, one reaction pathway for the pro duction of SiCl
2
is
SiCl
3
\$
SiCl
2
+ Cl
;
which quickly reacts with H
2
or Cl, leaving its steady-state concentration kinetically controlled, not equilibrium-
limited, and thus the observed pressure insensitivity. Past exp erimental studies have also shown that SiCl
2
concentrations remain constant across 1-100 Torr for a xed ratio of MTS/H
2
. It has b een obse rved that
only for
P <
0
:
1 Torr, the pro duction of SiCl
2
decreases with
P
due to reduced collision frequency.
Overall, the results in this section illustrate that the sensitivity of the mole fraction of various chemical
sp ecies under CVI-relevant conditions is highly dep endent up on the reaction pathways. If the pro duction of a
particular sp ecies is de p endent up on equilibrium-limited reaction or radical recombination, the sensitivity to
the pres sure is observed. Otherwise, the unimolecular pro duc tion of s p ecies is directly dep endent up on MTS
decomp osition, which, under the range of conditions considered here, shows no se ns itivity to the pressure
variation.
4.3. Eects of Temperature Gradient
Now, we examine the eects of temp erature gradient (
dT =dx
) on the decomp o sitio n pro ces s. We consider
temp erature variation (drop) of 10 K to 100 K across the reactor length,
L
, at dierent inlet temp eratures
(1150 K, 1250 K, 1350 K) at three values of op erating pressure (5 Torr, 52.5 Torr, and 100 Torr). These
conditions imply that the temp erature gradient,
dT =dx
, varies from ab out -190 K/m to - 19 K/m.
At
T
= 1150
K
, the decomp osition of MTS starts leading to a decrease in its value and the pro duction of
bypro ducts such as CH
4
(see Fig. 6(a) and (b)). However, the decomp osition/pro duction of MTS/CH
4
is en-
hanced due to the pre sence of a temp erature gradient. The rate of increase of the decomp osition/pro duc tion
of MTS/CH
4
tends to a constant value with an increase in
P
. Both M1 and M2 mechanisms yield the
same results, while the M3 mechanism fails to predict the decomp osition/pro duction of MT S/CH
4
at this
temp erature.
The enhanced decomp osition of MTS is due to the combined thermo dynamics and kinetics eects. A
hot inle t and cold outlet lead to a thermo dynamic driving of the endothermic decomp osition of MTS.
Additionally, the Arrhenius kinetics get accelerated near the hot inlet, thus enhancing the decomp osition.
The bypro ducts, such as CH
4
and HCl, are driven towards the co oler outlet zone, which also facilitates the
decomp osition of MTS in the hot zone. The pro duction of CH
4
is also enhanced with an increase in
j
dT =dx
j
due to combined thermo dynamics and kinetics eects. For example, after decomp osition of MTS in high
15


(a) MTS (
T
= 1150 K)
(b) CH
4
(
T
= 1150 K)
(c) MTS (
T
= 1250 K)
(d) CH
4
(
T
= 1250 K)
(e) MTS (
T
= 1350 K)
(f ) CH
4
(
T
= 1350 K)
Figure 8: Variation of mole fraction of MTS (a, c, e) and CH
4
(b, d, f ) with resp ect to the temp erature gradient at three values
of op erating pressure (5 Torr, 52.5 Torr, and 100 Torr) at inlet temp erature of 1150 K (a, b), 1250 K (c, d), and 1350 K (e,
f ). The results from the M3 mechanism are not visible in these gures as they are signicantly over-predicte d for MTS and
under-predicted for CH
4
.
16


temp erature zones into radicals such as CH
3
and the decomp osition of H
2
into H, CH
4
tends to form through
the radical recombination reaction, such as
CH
3
+ H
!
CH
4
:
(4.1)
Note that CH
4
is thermo dynamically stable at lower tem p eratures. Lastly, some of the comp eting reaction
pathways are suppressed due to the presence of a temp erature gradient. For example, without the presence
of a temp erature gradient, CH
3
can p olymerize to form C
2
H
6
, whereas with a temp erature gradient, CH
4
can b e rem oved rapidly to the low temp e rature region, leading to its enhanced pro duction.
At
T
= 1250 K and 1350 K, the de comp osition/pro duction of MTS/CH
4
is enhanced further, which is
exp ected as evident from Fig. 8. However, the dep endence of MTS and CH
4
mole fractions on 
T
changes
signicantly as the inlet temp erature is changed. At 1150 K, the decomp osition of MTS increase s with
T
gradually, while at 1250 K, a linear dep endence on
T
is observed, and at 1350 K, a rapid decomp osition of
MTS o ccurs even with low temp erature gradients. This is primarily due to kinetic eects. The dep endence
of CH
4
, although showing similar trends at 1150 and 1250 K, at 1350 K, its value tends to de crease with

T
in a weak manner.
The results from the M2 mechanism at
T
= 1250 K and 1350 K compare well with the M1 mechanism
for the pro duction of MTS. However, we can observe overprediction in the pro duction of CH
4
, particularly
at
T
= 1350 K, which als o shows sensitivity to
P
. As discussed b efore in Sec. 4.1, the pro duction of CH
4
by
the M2 mechanism showed overprediction compared to the M1 mechanism, which can b e attributed to the
absence of s ome of the reaction pathways, where CH
4
is involved, and the absence of s ome of the chemic al
sp ecies in this mechanism. Even at 1350 K, the results from M3 are signic antly inaccurate for b oth MTS
and CH
4
compared to the M1 mechanism.
Figure 9 shows the eects of variation in 
T
on two other bypro ducts, namely, SiHCl
3
and HCl. As the
qualitative variation of SiCl
4
and SiCl
2
with resp ect to the 
T
was observed to b e similar to SiHCl
3
and
HCl, resp ectively, the variation of mole fraction of these sp ecies is not shown here for the sake of brevity.
The dep endence of SiHCl
3
pro duction exhibits a complex b ehavior with resp e ct to 
T
at dierent values
of the inlet temp erature, which can b e attributed to the presence of multiple pathways of its formation and
the eects of kinetics and thermo dynamics. While at 1150 K, the pro duction of SiHCl
3
is enhanced, at
1350 K, it reduces with resp ect to 
T
. On the other hand, at 1250 K, SiHCl
3
pro duction is marginally
enhanced till 
T
ˇ
80 K, and afterwards a decrease in the pro duction is observed. The eect of pressure is
also evident, which leads to highe r levels of pro duction of SiHCl
3
at high pres sure due to the dominance of
recombination re actions .
The pro duction of HCl is enhanced with an increase in 
T
at all inlet temp eratures, although the
dep endence on 
T
shows dierent b ehavior at die rent inlet temp eratures. While at 1150 K, the rate of
increase of the pro duction of HCl with 
T
increases, at 1250 K, it stays nearly constant, and at 1350 K, it
tends to decrease. Furthermore, the eect of pressure is also evident, which shows a decrease in the pro duction
of HCl at higher pressure. The observed variation of pro duc tion of HCl with an increase in 
T
underscores
the role played by thermo dynamics, kinetics, and transp ort. For example, higher temp e rature leads to
enhanced decomp osition of MTS, which facilitates the pro duction of HCl. At lower temp eratures, HCl tends
to b e thermo dynamically stable, and thus, the re actions that consume HCl get limited. The temp erature
gradient also e ns ure s that the pro duced HCl in the hot zones is convected to the low temp erature exhaust
zone. Lastly, even though HCl pro duction is insensitive to pressure, low pressure enhances HCl evacuation
to low temp erature zones.
To summarize, the pre sence of a temp erature gradient favors the decomp osition of MTS, which in turn
leads to the pro duction of the bypro ducts. However, the eec t of op erating temp erature leads to a complex
dep endence of the mole fraction of the sp ecies on the temp erature gradient, which highlights the eects of
kinetics, thermo dynamics, and transp ort pro cesses, and the presence of multiple pathways for the formation
of several intermediate reac tive sp ecies.
4.4. Eects of Pressure Gradient
Now, we examine the eect of the pre ssure gradie nt on the de comp osition. To p erform this analysis, we
consider three inlet pressures, namely, 5 Torr, 52.5 Torr, and 100 Torr. For each inlet pressure, we vary the
pressure along the axial direction, leading to a favorable pre ssure gradient across the reactor, akin to a forced

ow condition. We sp ecify the pressure drop, 
P
, across the reactor length
L
to b e
 P
, with the value of

b eing varied from 10\% to 80\%. This implies that the pressure gradient in the axial direction (
dP =dx
) varies
17


(a) SiHCl
3
(
T
= 1150 K)
(b) H Cl (
T
= 1150 K)
(c) Si HCl
3
(
T
= 1250 K)
(d) H Cl (
T
= 1250 K)
(e) Si HCl
3
(
T
= 1350 K)
(f ) HCl (
T
= 1350 K)
Figure 9: Variation of mole fraction of SiHCl3 (a, c, e) and HCl (b, d, f ) with resp ect to the temp erature gradient at three values
of op erating pressure (5 Torr, 52.5 Torr, and 100 Torr) at inlet temp erature of 1150 K (a, b), 1250 K (c, d), and 1350 K (e, f ).
Some of the results from the M3 mechanism are not visible in these gure s as they are signicantly over- or under-predicted
compared to the M1 and M2 mechanisms.
18


(a) MTS
(b) CH
4
(c) SiHCl
3
(d) SiCl
4
(e) HCl
(f ) Si Cl
2
Figure 10: Variation of mole fraction of MTS (a), CH
4
(b), SiHCl
3
(c), SiCl
4
(d), HCl (e), and SiCl
2
(f ) with resp ect to the \%
drop in pressure at thre e values of op erating temp erature (1150 K, 1250 K, and 1350 K) at inlet pressure of 5 Torr.
19


from -7.7 Torr/m to -0.97 Torr/m at
P
= 5 Torr, from -80.8 Torr/m to -10 Torr/m at
P
= 52
:
5 Torr, and
from -153.4 Torr/m to 19.2 Torr/m at
P
= 100 Torr.
Figure 10 shows the eects of
dP =dx
on the mole fractions of the 6 chemical sp ecies at 5 Torr. Overall, we
do not observe sensitivity to
dP =dx
on the variation of any of the considered chemical sp ecies. At the inlet
pressures of 52.5 Torr and 100 Torr, similar dep endence of mole fraction on
dP =dx
is observed. Therefore,
results for these inlet pressures are not included here for the sake of brevity. The ma jor sensitivity is due
to the temp erature variation. As discussed in Sec. 4.2, the pressure se ns itivity was low for the ma jority
of the chemical sp ecies, which is also re
ected in the lack of sensitivity on the pressure gradient. Even
the intermediate reac tive sp ecies such as SiHCl
3
and SiCl
4
, which were found to b e sensitive to pressure
(see Fig. 7) do not show sensitivity to the considered values of
dP =dx
. These results imply that the MTS
decomp osition is primarily aected by the reaction mechanism and the kine tic s governing the pro cess.
While MTS, SiHCl
3
, HCl, and SiCl
2
show s imilar results with the M1 and M2 mechanis ms, the results for
CH
4
and SiCl
4
show dierences. In particular, the pro duction of CH
4
by the M2 mechanism is overpredicted
at T = 1350 K. The pro duction of SiCl
4
is under-predicted at b oth 1250 K and 1350 K. These results imply
the role of kinetics, which is more sensitive at higher te mp eratures and availability of the multiple pathways
for pro duction/consumption of bypro ducts and other reactive intermediates in the M1 mechanism.
4.5. Eects of Co mpositional Variation
Now, we examine the eects of comp ositional variation (

=
X
MTS
=X
H
2
) at the inlet by considering

2
(0
:
05
;
0
:
2). We consider two typ e s of reactor congurations. In the rst case, op erating temp erature and
pressure are sp ecied to b e 1250 K and 5 Torr, resp ectively. In the second case, the pres sure is sp ecied to b e
5 Torr, however, we c onside r a te mp erature gradient of 
T
= 60 K across the length of the reactor, with the
inlet temp erature sp ecied to b e 1250 K. These conditions are chosen based on the results discussed b efore,
where we observed minor sensitivity to pressure and ma jor sensitivity to temp erature and temp e rature
gradient. Figure 11 shows the results from the two typ es of cases. As the mechanism M3 yields inaccurate
results, we only consider M1 and M2 me chanisms to e xamine the eects of

.
In the case with te mp erature gradient, the decomp osition of MTS is enhanced, which is apparent from a
signicant decrease in the value of
X
MTS
for all values of

. The decrease is much higher for larger values of

.
Such b ehavior is due to the combined eect of higher temp erature and the presence of a temp erature gradient.
The enhancement in MTS decomp osition in the cases with higher tem p erature gradients is accompanied by
an increase in the pro duction of the intermediate reactive sp ecies. Furthermore, a higher level of pro duction
of these sp ecies o ccurs for larger values of

. While the variation of mole fraction of sp ecies such as SiHCl
3
,
HCl, and SiCl
2
with

is quasi-linear, the variation of
X
SiCl
4
tends b e nonlinear with resp ect to

in
b oth isothermal and temp erature gradient cases. Furthermore, the mole fraction of all sp ecies from the
M2 m echanism matches with the values obtained using the M1 mechanism, which again demonstrates the
adequacy of the M2 mechanism for dierent typ es of reactor conditions.
The re sults in this section de monstrate that the decomp osition of MTS can b e enhanced at a particular
op erating temp erature by enforcing a temp erature gradient across the reactor. Such an approach allows for
relatively low -temp erature conditions to still lead to enhanced decomp osition of MTS, which can subsequently
assist in SiC dep osition on the surface while accounting for the heterogeneous reactions. Note that the
commonly used op erating temp erature for the dep osition of SiC matrix comp osite is around 1200 K to 1300
K. Therefore, with a temp erature gradient across the reactor, the decomp osition pro cess can b e enhanced
further.
5. Conclusions
CVI is a reliable approach for fabric ating high-purity SiC matrix comp osite, which has excellent thermo-
mechanical prop erties. The quality of the fabricated comp osite by the CVI pro cess dep ends up on the
employed precursor, gas-phase decomp os ition, and heterogeneous surface reactions, which in turn are ae cted
by the op erating conditions. In this study, a PFR-based computational strategy is utilized to e xam ine the
eects of op erating conditions on the homogeneous gas-phase decomp osition pro cess while using MTS as a
precursor with H
2
as the carrier gas in a hot-wall cylindrical reactor at C VI-relevant conditions.
The decomp osition of MTS leads to the pro duction of seve ral bypro ducts and intermediate reactive
sp ecies, w hich, if not predicted well, c an lead to inaccurate results for the s ubse que nt heterogeneous re actions
and the asso ciated SiC dep osition. Therefore, we considered three chemical mechanisms with varying degree
of delity, namely, a detailed mechanisms (M1: 103 steps and 42 sp ecies), a mo de rate ly complex reduced
20


(a) MTS
(b) CH
4
(c) SiHCl
3
(d) SiCl
4
(e) HCl
(f ) Si Cl
2
Figure 11: Variation of mole fraction of MTS (a), CH
4
(b), SiHCl
3
(c), SiCl
4
(d), HCl (e), and SiCl
2
(f ) with resp ect to

for
isothermal and temp erature gradient conditions at inlet temp erature of 1250 K and pressure of 5 Torr.
21


mechanism (M2: 30 steps and 20 sp ecies), and a globally reduced mechanism (M3: 1 step and 5 sp ecies)
to demonstrate that certain numb er of steps and sp ecie s are needed to get accurate concentrations of the
sp ecies. While M1 and M3 mechanisms were considered from past studies, the M2 mechanism was derived
from M1 by using a reduction strategy, which ensured that it included the dominant reactions and their
sensitivity to key sp ecies for the range of op erating conditions considered in this study. Compared to the
detailed M1 mechanism, the computational cost of a PFR simulation by the M2 and M3 mechanisms was
found to b e 44.2\% and 26\%, resp ectively.
The eects of temp erature variation were e xam ine d for a range of temp eratures (1100 K to 1600 K) at
three values of pressure (5 Torr, 52.5 Torr, and 100 Torr). The results showed that decomp os ition of MTS
is enhanced b eyond 1150 K, and by 1350 K, it was c ompleted. T his was ac companie d by an increase in the
pro duction of CH
4
till ab out 1350 K, followed by a further de comp osition of CH
4
. The bypro ducts, such as
HCl and SiCl
2
showe d an increased pro duction till 1350 K, followed by a saturation. The other sp ecies, such
as SiHCl
3
and SiCl
4
showed a similar b ehavior where their pro duction showed an increase till ab out 1350 K,
followed by a decrease. These results highlighted the eect of kinetics , which showed signicant sensitivity
to temp erature variations.
The study of the eects of pressure variation was p erformed for a range of press ure s (5 to 100 Torr) at
three values of op erating temp erature (1150 K, 1250 K, and 1350 K). While MTS, CH
4
, HCl, SiCl
2
showed
no signicant sensitivity to pres sure , the sp ecies such as SiHCl
3
, and SiCl
4
showed an increas e in pro duction
with an increase in pre ssure, where the rate of increase showed an increase with temp erature. These results
demonstrated the sensitivity of the concentration of various chemical sp ecies to the reaction pathways. If the
pro duction of a particular sp ecies is dep endent up on equilibrium-limited reaction or radical recombination,
the s ensitivity to the pressure is observed. Otherwise, the unimolecular pro duction of sp e cies is directly
dep endent up on MTS decomp osition.
The study of the e ects of temp erature gradient (temp erature drop across the reactor) was p erformed at
three inlet temp eratures (1150 K, 1250 K, 1350 K) with a temp erature gradient of -190 K/m to -19 K/m.
The results s howed an increased decomp osition of MTS due to the presence of a temp erature gradient.
Howe ver, the eec t of inlet temp erature coupled with the presence of a temp erature gradient yielded a
complex dep endence of the concentration of the sp ecies on the sp ecied temp erature gradient. The study of
the eects of pressure gradient was p erformed at three values of inlet pressure (5 Torr, 52.5 Torr, and 100
Torr) and inlet temp erature (1150 K, 1250 K, and 1350 K) conditions. The results did not show sensitivity
to the pressure gradient for the considered conditions, even for some of the reactive intermediate sp ecies.
These results highlighted the complex inte rplay of kinetics, thermo dynamics , and transp ort pro cesses, and
the presence of multiple pathways for the formation of the reactive intermediate sp ecies and bypro ducts.
Based on the study of e ects of op erating temp erature and pressure conditions on the decomp osition of
MTS, we identied optimal conditions (1250 K and 5 Torr) and p erformed the study of eects of comp o-
sitional variation by varying

from 0.05 to 0.2 under isothermal and temp erature-gradient scenarios. The
variation of mole fraction of sp ecies such as SiHCl
3
, HCl, and SiCl
2
with

was found to b e quasi-linear
and the variation of
X
SiCl
4
tend to b e nonlinear in b oth isothermal and temp erature-gradient c ases. The
results also showed that the decomp osition of MTS can b e enhanced at a particular op erating temp erature
by enforcing a temp erature gradient across the reac tor.
The results from the mo derately complex M2 mechanism s howed go o d agreement with the detailed M1
mechanism under a wide range of op erating conditions, thus indicating that it can b e used for large-scale
simulations under CVI-relevant conditions. The discrepancies observed with the M2 mechanism for some
of the sp ecies at high temp eratures (greater than 1350 K) can b e attributed to the absence of ab out 50\%
sp ecies in comparison to the M1 mechanism and a signic antly lower numb er of re actions, which aects the
thermo dynamics and reaction pathways. The study further conrms the limitations of the globally reduced
single-step mechanism, such as the M3 mechanism, and e mphas izes the need for b etter reduction approaches.
Acknowledgments
This work was supp orted by the U.S. Department of Energy, Office of Science, Office of Basic Energy
Sciences (BES), Gas Phase Chem ical Physics (GPCP) through Grant \#DE-SC0024510.
References
[1]
 G. Meetham, journal of Materials Scie nc e 26 (1991) 853-860.
22


[2]
 R. E. Tressler, Comp osites Part A: Applied Science and Manufacturing 30 (1999) 429-437.
[3]
 M. Belmonte, Advanced engineering materials 8 (2006) 693-703.
[4]
 W. G. Fahrenholtz, E. J. Wuchina, W. E. Lee, Y. Zhou, Ultra-high temp erature ceramics: m aterials for
extreme environment applications, John Wiley \& Sons, 2014.
[5]
 Y. Bar-Cohen, High temp erature materials and mechanisms, volume 44, CRC Press Bo ca Raton, FL,
USA:, 2014.
[6]
 M. E. Levinshte in, S. L. Rumyantsev, M. S. Shur, Prop erties of Advanced Semiconductor Materials:
GaN, AIN, InN, BN, SiC, SiGe, John Wiley \& Sons, 2001.
[7]
 K. Hironaka, T. Nozawa, T. Hinoki, N. Igawa, Y. Ka toh, L. Snead, A. Kohyam a, Journal of nuclear
materials 307 (2002) 1093-1097.
[8]
 L. L. Snead, T. Nozawa , Y. Katoh, T.-S. Byun, S. Kondo, D. A. Petti, Journal of nuclear materials 371
(2007) 329-377.
[9]
 V. Presser, K. G. Nickel, Critical reviews in solid state and materials s ciences 33 (2008) 1-99.
[10]
 N. P. Padture, Journal of the Am erican Ceramic So ciety 77 (1994) 519-523.
[11]
 M. Mulla, V. Krstic, Journal of Materials science 29 (1994) 934-938.
[12]
 J. J. Cao, W. Chan, L. C . De Jonghe, C. J. Gilb ert, R. O. Ritchie, Journal of the American Ceramic
So ciety 79 (1995).
[13]
 R. Naslain, Philosophical Transactions of the Royal So ciety of London. Series A: Physic al and Engi-
neering Sciences 351 (1995) 485-496.
[14]
 K. M. Pre wo, American Ce ramic So ciety Bulletin 68 (1989) 395-400.
[15]
 T. Besmann, B. Sheldon, R. Lowden, D. Stinton, Science 253 (1991) 1104-1109.
[16]
 M. Wang, C. Laird, Journal of materials science 31 (1996) 2065-2069.
[17]
 S. Prouhet, G. Camus, C. Labrugere, A. Guette, E. Martin, Journal of the American Ceramic So ciety
77 (1994) 649-656.
[18]
 S. Zhu, M. Mizuno, Y. Kagawa, Y. Mutoh, Comp osites Science and Technology 59 (1999) 833-851.
[19]
 B. Liu, Y. Zhou, Comp osite Structures 307 (2023) 116610.
[20]
 Y. Xu, L. Cheng, L. Zhang, Carb on 37 (1999) 1179-1187.
[21]
 J. DiCarlo, H. M. Yun, G. Morscher, R. Bhatt, Handb o ok of ceramic comp osites (2005) 77-98.
[22]
 F. Sirieix, P. Goursat, A. Lecomte, A. Dauger, Comp osites science and technology 37 (1990) 7-19.
[23]
 A. Kohyama, M. Kotani, Y. Katoh, T. Nakayasu, M. Sato, T. Yamamura, K. Okamura, Journal of
Nuclear Materials 283 (2000) 565-569.
[24]
 A. Sayano, C. Sutoh, S. Suyama, Y. Itoh, S. Nakagawa, Journal of nuc lear materials 271 (1999) 467-471.
[25]
 J. Lamon, in: Handb o ok of ceramic c omp osites, Springer, 2005, pp. 55-76.
[26]
 C. P. Deck, H. Khalifa, B. Sammuli, C. Back, Sc ie nc e and Technology of Nuclear Installations 2013
(2013) 127676.
[27]
 Y. Katoh, K. Ozawa, C. Shih, T. Nozawa, R. J. Shinavski, A. Hasegawa, L. L. Snead, Journal of Nucle ar
Materials 448 (2014) 448-476.
[28]
 R. Liu, F. Wang, J. Zhang, J. Chen, F. Wan, Y. Wang, Ceramics International 47 (2021) 26971-26977.
23


[29]
 K. J. Probst, T. M. Bes mann, D. P. Stinton, R. A. Lowden, T. J. Anderson, T. L. Starr, Surface and
Coatings Technology 120 (1999) 250-258.
[30]
 D. P. Stinton, A. J. Caputo, R. A. Lowden, American Ceramic So ciety Bulletin 65 (1986) 347-350.
[31]
 R. R. Naslain, R. Pailler, X. Bourrat, S. Bertrand, F. Heurtevent, P. Dup el, F. Lamouroux, Solid State
Ionics 141 (2001) 541-548.
[32]
 B. J. Oh, Y. J. Lee, D. J. Choi, G. W. Hong, J. Y. Park, W. J. Kim, Journal of the Ame ric an Ceramic
So ciety 84 (2001) 245-247.
[33]
 T. No da, H. Araki, F. Ab e, M. Okada, Journal of nuclear materials 191 (1992) 539-543.
[34]
 R. Naslain, F. Langlais, G. Vignoles, R. Pailler, Mechanical prop erties and p erformance of engineering
ceramics I I: ceramic engineering and science pro ceedings 27 (2006) 373-386.
[35]
 A. Lazzeri, Ceramics and com p osites pro ces sing metho ds (2012) 313-349.
[36]
 D. A. Streitwieser, N. Pop ovska, H. Gerhard, Journal of the Europ ean Ceramic So ciety 26 (2006)
2381-2387.
[37]
 Z. Ramadan, I.-T. Im, Carb on letters 25 (2018) 25-32.
[38]
 C. M. Cha, D. Liliedahl, R. Sankaran, V. Ramanuj, Journal of the American Ceramic So ciety 105 (2022)
4595-4607.
[39]
 V. Ramanuj, R. Sankaran, B. Jolly, A. Schumacher, D. Mitchell, Journal of the American Ceramic
So ciety 105 (2022) 2421-2441.
[40]
 C. Deck, H. Khalifa, B. Sammuli, T. Hilsab eck, C. Back, Progress in Nuc lear Energy 57 (2012) 38-45.
[41]
 K. Dang, H. K. Chelliah, Internationa l Journal of Chemical Kinetics 54 (2022) 188-202.
[42]
 P. Mollick, R. Venugopalan, D. Srivastava, Journal of Crystal Grow th 475 (2017) 97-109.
[43]
 T. Ogawa, K. Fukumoto, H. Machida, K. Norinaga, Heliyon 9 (2023).
[44]
 G. D. Papasouliotis, S. V. Sotirchos, Journal of the Electro chemical So ciety 141 (1994) 1599.
[45]
 Y. Roman, J. Kotte, M. De Cro on, Journal of the Europ ean Ceramic So ciety 15 (1995) 875-886.
[46]
 S. Bammidipati, G. D. Stewart, J. R. Elliott Jr, S. A. Gokoglu, M. J. Purdy, AIChE Journal 42 (1996)
3123-3132.
[47]
 K. Norinaga, V. M. Janardhanan, O. Deutschmann, International Journal of Chemical Kinetics 40
(2008) 199-208.
[48]
 Y. Ge, M. S. Gordon, F. Battaglia, R. O. Fox, The Journal of Physical Chemistry A 111 (2007) 1462-
1474.
[49]
 Y. Ge, M. S. Gordon, F. Battaglia, R. O. Fox, The Journal of Physical Chemistry A 111 (2007) 1475-
1486.
[50]
 Y. Ge, M. S. Gordon, F. Battaglia, R. O. Fox, The Journal of Physical Chemistry A 114 (2010) 2384-
2392.
[51]
 S. H. Mousavip our, V. Saheb, S. Ramezani, The Journal of Physical Chemis try A 108 (2004) 1946-1952.
[52]
 J. Pe ng , B. Jolly, D. J. Mitchell, J. A. Haynes, D. Shin, Journal of the American Ceramic So ciety 104
(2021) 3726-3737.
[53]
 D. G. Go o dwin, H. K. Moat, R. L. Sp eth, Cantera: An ob ject-oriented software to olkit for chemical
kinetics, thermo dynamics, and transp ort pro ce sses,
http://www.cantera.org
, 2014. Version 2.1.2.
[54]
 R. Wang, R. Ma, Journal of crystal growth 310 (2008) 4248-4255.
24


[55]
 K. Guan, Y. Gao, Q. Ze ng, X. Luan, Y. Zhang, L. Cheng, J. Wu, Z. Lu, Chinese Journal of C he mical
Engineering 28 (2020) 1733-1743.
[56]
 J. J. Brennan, Materials Science and Engineering: A 126 (1990) 203-223.
[57]
 I. Golecki, Fib ers and Comp osites (2003) 112.
[58]
 T. S. Aarnˆs, E. Ringdalen, M. Tangstad, Scientic rep orts 10 (2020) 21831.
[59]
 M. Ksiazek, M. Tangstad, H. Dalaker, E. Ringdalen, Metallurgical and materials transactions E 1 (2014)
272-279.
[60]
 T. S. Aarnˆs, M. Tangstad, E. Ringdalen, Materials Chemistry and Physics 276 (2022) 125355.
[61]
 Z. Hu, D. Zheng, R. Tu, M. Yang, Q. Li, M. Han, S. Zhang, L. Zhang, T. Goto, Materials 12 (2019)
390.
[62]
 Y. Yang, W.-g. Zhang, Journal of Central South University of Technology 16 (2009) 730-737.
25

% Figures: Include as \includegraphics[width=\linewidth]{figureX.png} with captions

\end{document}
